\chapter{展望}
在最近十几年来, 类型论的研究有了大跨步的发展.
在理论方面, 我们对类型论的本质结构有了更加深刻的认识.
在实践层面, 有了许多重要的形式化与非形式化的工作.
不过, 仍然有许多问题有待回答.

利用基于类型论的定理证明检查器形式化了许多著名的成果,
这足够证明类型论形式化数学的潜力.
类型论的核心部分往往非常简洁, 使得证明其性质相对方便,
但是实际使用时会使得写出的证明无比繁琐.
在这个方向, 我们仍然需要进一步改进定理证明的\textbf{前端},
即与使用者直接交互的部分.
例如, 需要让定理证明的体验与数学家的习惯相契合,
能让用户使用各个数学领域常用的符号和缩写,
组织证明时能流畅使用各种常见的证明技巧,
平凡的计算细节可以自动化处理, 等等.
2023 年发表的 Sebastian Ullrich~\cite{ullrich:2023:frontend} 的学位论文可以参考.

与此相关, 基于神经网络的人工智能在定理证明中也有应用的潜力,
例如在繁杂的定理库中快速筛选出与当前目标相关的部分.
由于已经有成熟的组件检查输出的证明是否正确,
这就避开了最近出现的大语言模型在关于事实准确性方面的短板.
有不少工作将神经网络整合到定理自动证明体系中,
如 LeanDojo~\cite{yang:2023:leandojo} 等.

有一类类型论在近年来得到了许多关注, 它们称为\textbf{模态类型论}.
这起源于逻辑学中的模态逻辑.
例如可以用模态 \(\lozenge p\) 表示“可能 \(p\)”,
进而研究关于可能性与必然性的逻辑.
模态逻辑可用于研究可能性、时态、知识、道德等等
在哲学与逻辑学中有意义的话题.\footnote{例如考虑这个有趣的悖论:
我认识张三, 我不认识面前这个戴面具的人, 因此张三和面具人不相同.}
而在数学方面, 许多重要的概念都可以统一到类型论中的模态框架下,
同时一些综合数学系统中也会遇到适合用模态公理化的对象.
模态类型论可能可以将各种优势各异的类型系统统一到同一个类型论中.

在同伦类型论方面, 我们知道简单类型论与积闭范畴之间有等价关系;
而 \cite{clairambault:2014:biequivalence} 中证明了
带有外延相等类型的 Martin-L\"of 类型论与局部积闭范畴之间有等价关系.
那么内涵 Martin-L\"of 类型论对应什么呢?
一个合理的猜想是它与 \((\infty, 1)\)-局部积闭范畴等价.
但是证明此事困难重重. 第一在于我们尚未了解清楚 Martin-L\"of 类型论
构成怎样的 \(\infty\)-范畴, 或许这需要合适的 \(\infty\)-自然模型的概念.
其次是对于二者之间的等价也尚不知道如何构造,
例如从 \((\infty, 1)\)-局部积闭范畴得到类型论,
目前只知道局部可表范畴的情况.
而再进一步, 同伦类型论对应什么呢?
应当是某种初等 \((\infty, 1)\)-意象的概念.
但是这连定义都尚未明确, 更不用提证明了.

立方类型论解决了同伦类型论的计算问题.
例如 Brunerie~\cite{brunerie:2016:number} 证明了
存在自然数 \(n\) 使得 \(\pi_4(\mathbb S^3) = \mathbb Z/n\mathbb Z\).
因为立方类型论中任何一个自然数表达式都判值相等于某个具体的自然数
\(\cons{succ}(\cons{succ}(\dots (\cons{zero})))\),
因此理论上我们可以直接在立方类型论中化简, 最后应当得到
\((2, p)\), 其中 \(p\) 是同构的证明. 这样就不再需要额外证明
这个自然数 \(n = 2\). 然而, 因为所需的计算资源与时间过长,
目前还没有电脑成功完成这个计算.
目前有一些旨在改进立方类型论的计算效率的工作,
但是也可能需要设计全新的类型论才能解决这个问题.

关于泛等数学基础, \cite{ufp:2013:hottbook}
已经发展了许多半形式化的基于泛等数学基础的理论.
我们需要更多的在泛等基础中半形式化或者非形式化的数学工作.
并且这些工作不应当限制于类型论的话题中,
而是应当更加积极地探索泛等数学基础能为更广泛的领域带来怎样的贡献.
非形式化的数学中, 可以不提及或者模糊处理类型的概念.
然而, 同伦类型论, 包括立方类型论,
是目前泛等数学基础唯一的严格形式化方法.
泛等数学基础是否只能在类型论中实现呢? 这个问题尚待解答.

