\chapter{起源}
\section{Russell的类型论} %(1900 -- 1927)
1899年, Zermelo 发现了 Russell 悖论,
但是他并没有发表这个结论.
1901年这个悖论由 Russell 独立发现.
它的论证非常简洁.
\begin{theorem}[Russell]
以下说法不可能成立: 任给一个含有变量 \(x\) 的命题 \(P(x)\),
都有一个集合 \(y = \{x \mid P(x)\}\), 使得
\(z \in y\) 当且仅当 \(P(z)\) 成立.
\end{theorem}
\begin{proof}
\(P(x) = (x \notin x)\) 是含有变量 \(x\) 的命题.
而如果 \(R = \{x \mid x \notin x\}\) 是集合,
那么 \(z \in R\) 当且仅当 \(z \notin z\).
特别地, \(R \in R\) 当且仅当 \(R \notin R\).
矛盾.
\end{proof}
因此, 我们习以为常的集合记号 \(\{x \mid P(x)\}\) 是
直接导致矛盾的. 这个悖论急迫要求数学家重新审视数学和逻辑
的基础. 在这个基础上 Zermelo 等提出并逐渐完善的 ZFC 集合
论, 大家或许已经有所了解. 但是 Russell 本人提出的解决
方案则走了另一条思路. 这里我们介绍一个由 Alonzo Church 简化的版本.%
~\cite{sep:2022:typetheory}

我们可以发现, 数学中我们根本不会需要讨论 \(x \in x\) 这样的
命题. 实数 \(x\) 属于实数集 \(\mathbb R\), 而一些实数的
集合 (如开集) 构成集合族 \(\Omega\), 我们偶尔还会讨论集合族
的集合, 如所有超滤的集合 \(\beta X\) (这在拓扑学的
Stone 表示定理中有提到). 注意这些数学对象之间都是有明确的层级的,
这一级的对象属于上一级的对象, 层级混乱的命题一般没有数学意义.
更严格来说, 对象之间是有分类的. 譬如虽然二维的点 \((\pi, 3)\)
和三维空间的点 \((0,1,-1)\) 都是同一个层级的, 但是我们也基本不会
把它们放在同一个集合中.

从这个观察出发, 我们可以把这种分类的概念严格化. 我们设当前研究
的最基本的个体的分类称作 \(\iota\). 如我们研究数论的时候
\(\iota\) 代表全体自然数, 研究分析学的时候就是全体实数, 等等.
(对于更复杂的理论, 完全可以有多个基本个体的分类, 如平面几何学可能
把点、直线、圆各分一类.) 这里我们不妨就考虑是自然数的情况.
其次我们需要有一类数学对象表示命题, 因此我们把所有命题构成的分类
称作 \(o\).

其次, 我们需要从基本的分类出发组成更复杂的分类. 譬如说可以用
\(\iota \to \iota\) 表示自然数到自然数的函数. 为什么选择
函数而不是集合呢? 注意要描述 \(A\) 的一个子集, 就等同于描述
各个元素是否属于这个子集. 换句话说, 我们给每个元素赋予一个命题.
因此 \(\iota \to o\) 恰好就是自然数的子集的分类. 这样,
集合就可以看成是函数的特殊情况.

多元函数如何呢? 我们可以再引入一个代表集合Descartes乘积的分类.
但是由数学家 Moses Sch\"onfinkel 提出了一种方便
的技巧. 注意从 \(m\) 元集到 \(n\)
元集的函数恰好有 \(n^m\) 个. 而 \((n^m)^p = n^{mp}\).
而集合也有类似的等式: 值域是函数的函数 \(A \to (B \to C)\)
和二元函数 \(A \times B \to C\) 是一一对应的. 这其实在
数学中出现过许多次. 譬如向量空间 (或一般来说, 模) 的张量积满足
\(A \otimes B \to C\) 和 \(A \to (B \to C)\) 是同构的.
带点拓扑空间上也有 \(A \wedge B \to C\) 和 \(A \to (B \to C)\)
同构, 这里 \(\wedge\) 是压缩乘积 (smash product), 而
\(\to\) 是连续函数空间. 因此, 我们不需要额外引入
多元函数的概念. 这个技巧由 Haskell Curry 的使用而受到推广,
因此今天我们将其称作\textbf{Curry化} (currying).

当然, 这只是一套能实现所有功能的方案. 我们也可以选择其他的组合.
譬如规定 \((s_1, s_2, \dots, s_n), n\in\mathbb N\) 为
\(n\) 元关系的分类, 即含有 \(n\) 个变量的命题组成的分类.
其中第 \(i\) 个变量的分类是 \(s_i\). 这样的话,
\(()\) 就是命题的类型, 而函数则在建立起逻辑之后定义为特殊的
二元关系.

这些分类, 就是我们所称的类型. 这个类型论 (以及其变体) 被称为%
\textbf{简单类型论}. 当然, 只有类型还不行, 我们需要能
谈论类型里的元素. 这里需要注意的是, 元素并不能独立于类型而存在.
我们不能先假设有一个元素, 然后讨论它属于什么类型. 这在集合论中
是一样的, 我们不能假设有一个物体, 然后讨论它是不是集合, 因为
一切都是集合.\footnote{实际上确实可能有这种语言, 但严格来说
这就和我们现代使用 “设 \(\epsilon\) 是任意小的正实数” 一样,
是严谨用语的一种缩写简化.} 在类型论中一切都首先归属于某个类型, 而后才能开始
讨论. 我们把 “\(E\) 的类型是 \(s\)” 写作 \(E : s\).

在简单类型论中, 构造和使用元素的方法也非常简单. 首先, 对于函数类型,
我们可以用数学中常见的方法构造一个函数 --- 写出其解析式. 这种
方法早在Euler的时代就已经熟知了:
\[x \mapsto \sin x^2 + 1.\]
这就规定了一个函数. 在逻辑学中, 我们更习惯用这样的符号:
\[\lambda x. \sin x^2 + 1.\]
不过这只是具体记号的细微差别. 更严格的说, 如果假设 \(x\) 是
\(s_1\) 类型的变量, 有一个含有 \(x\) 的表达式 \(E\) 是
\(s_2\) 类型的; 那么 \(\lambda x. E\) 就是 \(s_1 \to s_2\)
类型的表达式.

反过来, 我们有了一个函数, 要如何使用它呢? 唯一的方法当然是
代入某一个的自变量. 因此如果 \(F : s_1 \to s_2\), 并且
\(A : s_1\), 那么 \(F(A) : s_2\). 这里因为我们经常会
使用函数, 我们把 \(a \to (b \to c)\) 简写为 \(a \to b \to c\),
即默认是右结合的 (减法则是左结合的: \(a - b - c = (a - b) - c\)).
同时, 我们省略函数求值的括号, 并且默认左结合: \(FAB = (F(A))(B)\).

除了函数, 我们还有命题. 这时候我们就可以引入熟悉的逻辑结构了.
譬如说 “且” 命题就是 \(\wedge : o \to o \to o\).
如上面所述这里表示一个二元函数. 而 “对于所有” 就是
\(\forall : (s \to o) \to o\). 譬如说我有
一族关于 \(n\) 的命题 \(n^2 \ne 2\),
那么 \(\forall (\lambda n. n^2 \ne 2)\)
就接受这一族命题作为参数, 而产生一个命题, 表示 “对于所有的
\(n\) 这些命题都成立”. 还有一个非常重要的东西是
\(\epsilon : (\iota \to o) \to \iota\). 如果
\(P : \iota \to o\) 是一族命题, 并且存在一个对象满足
这个命题, 那么 \(\epsilon(P)\) 表示某一个满足条件的对象.
有些版本中需要把这里的 “存在” 改成 “存在唯一”, 因为
这里实际上默认了选择公理.

我们这里说的, 譬如 \(\wedge\) 表示且命题, 是靠添加逻辑公理
来保证的. 这和集合论是类似的: 集合论除了集合的公理之外,
也是默认了所有
的逻辑公理的. 如 “如果证明了 \(p \Rightarrow q\), 并且证明了
\(p\), 那么就可以证明 \(q\)”, 或者 “如果已经证明了 \(\exists
(\lambda n. P(n))\), 那么 \(\epsilon(P)\) 就满足 \(P\),
或者写成 \(P(\epsilon(P))\)”.
这些公理不涉及具体的数学对象.

至于那些涉及具体数学对象的公理, 就称为非逻辑公理. 譬如说在
研究数论的时候, 我们就会规定一个元素 \(0 : \iota\),
并且有一个函数 \(S : \iota \to \iota\) 表示自然数的后继.
那么 \(S(S(S(0)))\) 就表示 \(3\). 我们还会添加数学归纳法
作为公理:
\[\forall (\lambda P. P(0) \wedge
\forall(\lambda n. P(n)\Rightarrow P(S(n)))
\Rightarrow \forall (\lambda n. P(n))).\]
具体公理的细节比较乏味, 这里就省略了. 读者可以参考\cite{farmer:2008:virtues}.
据笔者的阅读它关于命题逻辑的公理是有瑕疵的, 但是无伤大雅.

人们依据这种形式系统, 写出了计算机程序 Isabelle~\cite{tobias:2002:isabelle},
用计算机的准确性验证数学证明. 已经有不少数学理论被 Isabelle 形式化.
譬如质数定理~\cite{eberl:2018:pnt}. 这说明类型论与集合论一样,
有形式化数学的基本能力.

\section{标准语义与Henkin语义}\label{beginning:henkin}
%为什么叫做高阶逻辑
简单类型论常常被称作\emph{高阶逻辑}. 这是相对于一阶逻辑
说的. 一阶逻辑中一切的 \(\forall, \exists\) 量词, 其
作用的变量都是取值于同一个论域的. Peano 公理中这个论域
就是自然数, ZFC 集合论中这个论域就是集合. 因此, Peano公理
对于归纳法做了特殊处理: 归纳法说的是 “对于所有含变量
\(n\) 的\emph{命题}\footnote{含有变量的命题称作\textbf{谓词},
有 \(k\) 个变量就称为 \(k\)元谓词.}
\(P(n)\), 如果 \(P(0)\) 成立,
并且 \(\forall n. P(n) \Rightarrow P(n+1)\), 那么
就有 \(\forall n. P(n)\).” 这里出现了 \(\forall P\)
的表达. Peano 公理系统在一阶逻辑中无法做出这个表达, 因此
实际上并不是添加了一条公理, 而是无数多条, 对于每一种
可能的 \(P(n)\) 都添加了一条独立的公理. 如果某个逻辑系统
允许这样的量词, 就称之为\textbf{二阶逻辑}. 这里
的命题含有谓词变量 \(P\), 也就是关于谓词的谓词.
进一步, 二阶逻辑允许讨论关于谓词的谓词, 那么就会
有关于“关于谓词的谓词”的谓词. 这就是三阶逻辑.

简单类型论中, 由于 \(\forall : (s \to o) \to o\) 中
\(s\) 是任意的类型, 如果 \(s = \iota\), 则这是一阶逻辑
的量词; 如果 \(s = (\iota \to o)\), 那么这就是二阶量词;
以此类推. 因此简单类型论包含了任意高阶的逻辑, 从而有非常
强的表达力.

%语义, Henkin model
与其它的高阶逻辑一样, 简单类型论有两种语义, 标准语义与Henkin语义.
其中前者是后者的特殊情况.

\begin{definition}
简单类型论的一个\textbf{Henkin 模型} \(M_\bullet\) 满足
\begin{itemize}
\item 对于每个类型 \(s\), 选择一个集合 \(M_s\);
\item 对于函数类型 \(s_1 \to s_2\), 选择的集合 \(M_{s_1 \to s_2}\)
是函数集 \(M_{s_1} \to M_{s_2}\) 的子集.
\end{itemize}
\textbf{标准模型}中要求 \(M_{s_1 \to s_2} =
(M_{s_1} \to M_{s_2})\). 这些集合的选取需要使得每条
公理都成立.
\end{definition}
因为语言是有限长的, 至多可以表达出可数个\(\mathbb N \to \mathbb N\)函数,
绝大部分函数都是无法表达的. 因此 Henkin 模型更加灵活,
可以不包含不需要的函数, 在证明相关性质的时候更加有用.
每个命题在Henkin模型中都有一个真值. 对于了解逻辑学的
读者, 这些事情应该不算新奇.

注意模型是在形式系统
的\emph{外部}进行的. 换句话说, 我们现在正在研究这个
形式系统, 而不是使用这个形式系统; 因此我们研究时使用的形式
系统可以任意选取, 比如选择ZFC. 因此命题的真值也都是
在这个外部的形式系统(也叫\textbf{元逻辑})中得到的.
这个真值选取的正确性由一条定理保证:
\begin{theorem}[Henkin可靠性]
如果某个命题在系统中可证, 那么它在所有Henkin模型中为真.
\end{theorem}
一个重要的应用就是其逆否命题:
如果构造了某个Henkin模型, 其中我们
关心的命题为假, 那么在系统中就不可证明这个命题. 注意“不可
证明”与“可证明其否定”的区分. 类似的, 如果某个Henkin模型中
我们关心的命题为真, 那么在系统中就不可证伪这个命题.

反过来的性质通常更难证明, 但是更加有用:
\begin{theorem}[Henkin完备性]
某个命题在所有Henkin模型中都为真, 那么这个命题可证.
\end{theorem}
我们在 \ref{intro:semantics}~节已经提到过这两个性质.

\section{\texorpdfstring{\(\lambda\)}{Lambda}-演算与组合子逻辑}
\subsection{\texorpdfstring{\(\lambda\)}{Lambda}-演算}\label{beginning:lambda}
上面我们介绍了简单类型论. 而简单类型论中最核心的内容
其实是关于如何形成和使用函数. 我们不妨把这部分内容抽象出来
进一步研究其性质. 这一部分工作是由 Church 从他提出的类似的
一套数学基础\cite{church:1932:untyped}中进一步提取出来
的; 在\cite{church:1936:lambda}中他研究了 \(\lambda\)-演算
的递归论性质.

剥去其他无关的细节后, 我们得到的系统尤其简洁. 它只有变量,
函数构造 \(\lambda x. M\), 以及函数求值 \(M(N)\).
它满足
\[(\lambda x. A)(B) = A[x/B].\]
其中 \(A[x/B]\) 表示将 \(A\) 中包含的 \(x\) 变量全部
代换成 \(B\). 这个等式称作 \textbf{\(\boldsymbol\beta\)-相等}.
正如我们在前言中所述, 这个描述看似简单, 但是变量的处理
会引起一些麻烦, 在这里我们不做处理. 这个等式描述了构造一个函数,
而后马上对它求值, 会得到什么. 有时候我们也可以反过来考虑,
对一个函数求值, 然后以此构造一个函数, 会得到什么呢?
\[\lambda x. f(x) = f.\]
因为这实际上只是把原先 \(f\) 代表的函数包装了一下, 并没有
改变它的实际性质, 因此我们认为这仍然是等于 \(f\) 的.
这个等式称作 \textbf{\(\boldsymbol\eta\)-相等}.
当然, 这里也有变量处理的麻烦: \(f\) 的表达式中不能含有 \(x\) 本身.

仅仅有这些东西, 我们就能构造出一套非常丰富的结构了. 譬如说,
我们可以编码自然数: 将自然数 \(n\) 编码为
\[\lambda f. \lambda x. \underbrace{f(f(f(...(}_{n\text{个}}x))))\]
它的意义是 “把函数 \(f\) 迭代 \(n\) 次”, 用迭代次数
来编码自然数. 那么我们可以立刻实现加法:
\[\cons{add} = \lambda m. \lambda n. \lambda f. \lambda x. mf(nfx).\]
这是什么意思呢? 给定两个自然数 \(m, n\) 以及一个函数 \(f\),
它先把 \(f\) 迭代了 \(n\) 次得到 \(nfx\) (因为 \(n\) 的定义
就是把函数迭代 \(n\) 次的函数), 然后在此基础上又将
\(f\) 迭代了 \(m\) 次. 因此这就编码了两个自然数的加法.

乘法也是类似的, 我们有了一个函数 \(g\) 等于 \(f\) 的 \(n\)
次迭代后, 我们可以把 \(g\) 进行 \(m\) 次迭代, 就得到了
\(f\) 的 \(m\cdot n\) 次迭代:
\[\cons{mul} = \lambda m. \lambda n. \lambda f. \lambda x. m(nf)x.\]
这里连着的 \(\lambda\) 已经有些多了, 我们把这些简写成
\(\lambda mnfx. m(nf)x\), 即只用一个 \(\lambda\) 符号.
读者可以自行构造指数函数 \(m^n\). Church 在论文~\cite{church:1936:lambda}中,
进一步构造了判定自然数是否相等的函数, 从而能够在 \(\lambda\)-演算
中编码 Diophantus 方程. 而在\cite{church:1932:untyped}中,
他与上面类似地添加了逻辑运算, 从而得到了逻辑的基础.

\subsection{组合子逻辑}
另一方面, 在1920年左右, 数学家 Moses Sch\"onfinkel 提出了\textbf{组合子逻辑}~\cite{schonfinkel:1924:combinator}.
上面我们提到的简单类型论, 以及我们熟知的一阶逻辑等等, 其中
一个最基本也是最容易被忽视的概念是\emph{变量}. 在前言
中, 我已经说明了处理变量的一部分困难. Sch\"onfinkel 试图
找到一种方式, 能够不使用变量而描述函数, 进而给出逻辑的更方便
的定义. 原始论文中用的具体字母有些不同, 但是其余内容和我们
下面介绍的是一致的.

既然我们关注的是如何不靠变量表达一切函数, 那么我们自然要先把
具体的函数, 如 \(\log, \sin\) 等等先放到一边, 关注函数
运算本身的抽象结构. 我们语言中需要有函数求值 \((xy)\), 把两个表达式拼合
成更大的表达式. 因为函数求值 \(f(x)\) 本质上是个二元运算,
所以这里和乘法写得像一些: \((fx)\), 强调其运算属性. 因为
这里所有表达式都将不再有变量, 我们称他们为\textbf{组合子}.
注意这里的 \(x,y\) 等等都是代表一个完整的表达式. 它们并不是
语言本身的变量, 而是为了描述语言而引入的记号. 这我们称为
元变量.

首先我们有恒等函数: 它满足
\[(\cons{I}x) = x.\]
这很好理解. 那么譬如有 \((\cons{I}\cons{K}) = \cons{K}\),
等等.

其次, 我们有常数函数: 对于任何表达式 \(x\), 我们都希望有一个
表达式, 它代表恒等于 \(x\) 的常函数. 换句话说, 如果这个表达式
是 \(K_x\) 的话, 那么我们希望有等式
\((K_xy) = x\).
因此我们引入第一个原始组合子 \(\cons{K}\), 使得对于任何
\(x\), \((\cons{K}x)\) 就是满足要求的常函数. 换句话说,
\[((\cons{K}x)y) = x.\]
这里括号已经开始有点多了. 我们和上面一样, 规定函数求值运算
是左结合的. 上面的表达式也可以写成 \(\cons{K}xy\),
表达相同的意思.

我们再引入函数复合: 给定两个函数 \(f, g\), 我们需要能表述
其复合 \(h\), 满足 \(hx = f(gx)\). 因此我们引入组合子 \(\cons{B}\),
使得 \(\cons{B}fg\) 就是其复合. 换句话说,
\[\cons{B}fgx = f(gx).\]

再者, 对于一个二元函数 \(fxy\), 我们希望能把它的第二个参数
代入 \(gx\). 换句话说, 我们希望找到函数 \(h\)
使得 \(hx = fx(gx)\). 因此我们引入组合子 \(\cons{S}\),
使得 \[\cons{S}fgx = fx(gx).\]

这些东西有些无穷无尽了. 对于我们所希望的一切函数
操作, 能不能只由有限个组合子互相组合表达呢? Sch\"onfinkel
接下来指出, 只需要 \(\cons{S}, \cons{K}\) 两者就足够
表达剩余的所有操作. 举个最简单的例子, 如何用它们表示 \(\cons{I}\)
呢? 请看:
\[\begin{aligned}
  &\quad~\cons{S}\cons{K}\cons{K}x\\
\text{(\(\cons{S}\) 的定义) }&= \cons{K}x(\cons{K}x)\\
\text{(\(\cons{K}\) 的定义) }&= x.
\end{aligned}\]
换句话说, \(\cons{S}\cons{K}\cons{K}\) 这个函数,
在任何表达式 \(x\) 处求值的时候得到的都是 \(x\) 自身.
所以这个函数就是恒等函数! 关于这些组合子的更多性质,
读者可以参阅这本十分有趣的逻辑谜题书~\cite{schonfinkel:1924:combinator}.

这些组合子在 1927 年被 Haskell Curry 重新发现.
后来, 它们被计算机科学借鉴, 用来作为程序运行的一种简洁而
完备的模型. 事实上, 当代计算机中有一类编程语言, 称为
函数式语言, 它们在编译成程序的时候就经常会使用组合子作为
中间步骤.

有了这些组合子, 我们就可以加入具体的函数来表达丰富的数学对象了.
对于逻辑来说, 我们可以加入组合子 \(\forall\), 使得
\(\forall A\) 表示 (在常规的有变量的语言中的) \(\forall x. A(x)\).
作为例子, 现在我们可以重新表述我们在一阶逻辑中一条常见的公理:
\[(\forall x. A(x)) \implies A(c),\]
其中 \(c\) 是某个常数. 在组合子逻辑中, 这就可以写作
\[\cons{Im}(\forall A)(A c)\]
其中 \(\cons{Im}\) 组合子接受两个参数 \(p, q\), 表示命题
\(p \Rightarrow q\). 这样, 我们就完全避免了需要严格描述诸如
“将 \(x\) 代换为 \(c\)” 之类的麻烦. 这里再度提醒读者,
\(A, c, p, q\) 在这里都是元变量, 只有 \(x\) 是真正的语言内
变量.

\subsection{Curry悖论}
然而, 把组合子或者 \(\lambda\)-演算直接用于逻辑基础是
不可行的. 首先由 S. C. Kleene 与 J. B. Rosser 在 1935 年提出了
组合子逻辑的悖论; 其次这个悖论被 Curry 简化. 为了理解这个悖论, 我们首先来
看一个数学中经常会遇到的问题: 不动点问题.

假设我有一个任意的函数 \(f\), 我希望求解方程 \(x = f(x)\).
事实上, 递归函数也能这么定义. 假如我有递归定义 \(f(x) = M\),
其中 \(M\) 是某个含有 \(f\) 的表达式, 那么我们就能写成
\(f(x) = g(f,x)\). 换言之, 我们需要找到 \(f = \lambda x. g(f,x)\)
的解; 再重写一下, 就是需要求 \(G(f) = \lambda x. g(f,x)\)
的不动点. 由此可见, 能够求解不动点问题, 是语言表达力强的体现.
这里就体现了 \(\lambda\)-演算的强大能力.

\begin{theorem}
对于任何函数 \(f\),
存在表达式 \(M\) 满足 \(M=f(M)\).
\end{theorem}
\begin{proof}
取
\[M = (\lambda m. f(mm))(\lambda m. f(mm))\]
根据\(\beta\)-相等的等式 \((\lambda x. A)B = A[x/B]\),
代入就得到
\[M = f(mm)[m/\lambda m.f(mm)]
= f((\lambda m. f(mm))(\lambda m.f(mm)))
= f(M).\]
恰好满足要求.
\end{proof}
这种递归的技巧的确非常天才, 读者可以再仔细研究这个表达式,
揣摩产生它的动机. 既然对于任何一个 \(f\), 这个表达式都是
其不动点, 那么我们自然可以提取出一个映射来:
\[\cons{fix}(f) = (\lambda m. f(mm)) (\lambda m. f(mm)).\]
换句话说, \(\cons{fix} = \lambda f. (\lambda m. f(mm)) (\lambda m. f(mm))\)
就是\textbf{不动点组合子}.\footnote{虽然这并不是组合子演算的
表达式, 但是广义来说, 我们把所有不含自由变量
的 \(\lambda\)-表达式都称作组合子.}
上一节中我们说过, 组合子演算可以表达所有的函数运算, 这个也
不例外. 它对应的组合子为
\[\cons{Y} = \cons{S}(\cons{K}(\cons{S}\cons{I}\cons{I}))
(\cons{S}(\cons{S}(\cons{K}\cons{S})\cons{K}))
(\cons{K}(\cons{S}\cons{I}\cons{I})).\]
这个复杂的表达式则是由将 \(\lambda\)-演算和组合子演算
机械地互相转换的算法得到的.

读到这里, 谨慎的读者可能有疑问了: 万一某个表达式没有不动点
怎么办? 如果有多个不动点, 算出来的会是哪一个呢? 我们来验证几个
具体的例子.

首先对于常函数 \(K_x(y) = x\), 我们计算
\[\cons{fix}(K_x) = K_x(\cons{fix}(K_x)) = x.\]
因此它计算出的的确是其唯一的不动点. 而对于恒等函数
\(\lambda x. x\), 一切都是其不动点. 计算得到
\[\cons{fix}(\lambda x.x) = (\lambda x.xx)(\lambda x.xx).\]
我们记右边两个重复的表达式为 \(\omega\), 令 \(\Omega = \omega\omega\).
那么能不能继续把这个表达式的具体值求出来呢? 读者如果自行试图
继续化简的话, 就会发现无论怎么化简, 都会立刻兜圈回到原本
的表达式. 换句话说, 这种表达式的求值操作是\textbf{不停机}的.

在逻辑中, 最简单的不存在不动点的函数是什么呢? 当然是命题的否定!
换句话说, 如果我们记 \(\cons{not}\) 接受一个命题, 表示
其否定, 那么 \(\cons{fix}(\cons{not})\) 就等于其自己的否定,
从而导出了矛盾. 这就是 Curry 悖论.

\section{简单类型 \texorpdfstring{\(\lambda\)}{Lambda}-演算的典范化}
\label{beginning:ccc}
\begin{center}\fbox{\color{gray!80!blue}
\parbox{0.8\textwidth}{%
\textbf{关于译名.} 对于已经了解这部分内容的读者, 有必要
在这里澄清相关术语的翻译问题. 在重写系统(rewriting system)下
不可规约(reduce)的表达式称为既约形式(normal form);
有强/弱停机性(strong/weak normalization)的概念.
而在类型论中, 一般而言有典范形式(normal form), 无变量时
称作闭典范形式(canonical form). 注意英文中 normal form
有不同的含义, 我们翻译为不同的中文词汇.
}}%
\end{center}

我们上面已经看到, 不停机的 \(\lambda\)-演算或组合子
很难成为逻辑的载体. 不过 Russell 的简单类型论目前来看还
没有受到影响: 诸如 \(\cons{fix}\) 的表达式在简单类型论
中是不符合类型规则的. 那么对于添加了类型的演算, 能不能保证
它不出现类似的问题呢? 在这个时期, 诞生了很多类型论以及对应
研究它们的技术.简单类型组合子或 \(\lambda\)-演算的相关
性质可以用初等组合学的手段加以证明\cite{loader:1998:stlc},
但是我们这里选择更加展示其结构的一种思路\cite[\S4.2]{sterling:2021:thesis}.

首先, 我们想要的性质的精确定义是什么? Sterling
在 \cite[\S5.1]{sterling:2021:thesis} 中有非常
精当的论述.

我们最常见到的概念是\textbf{停机性}. 使用
有向图来描述“化简”(术语是\textbf{规约})的方向, 我们
可以定义不能进一步化简的表达式为\textbf{既约形式}. 如果
对于每个表达式, 都存在一条路径达到既约形式, 则称这个有向图
代表的规约系统是\textbf{弱停机}的; 如果所有的路径都必须
在有限步内达到既约形式, 则称规约系统\textbf{强停机}.
假如从同一个表达式出发的任意两条规约路径, 都可以延长至共同
的终点, 则称这个规约系统有\textbf{合流性}(confluence).

具体到类型论中, 我们可以规定一些规约的规则. 与上面抽象地
使用有向图表示规约不同, 这里的规则可以对任何子表达式使用.
例如有规则 \(1 + 1 \rightsquigarrow 2\), 那么就可以
得到 \((1+1)+3 \rightsquigarrow 2 + 3\). 这样, 我们
得到的就是一套\textbf{重写系统}. 我们可以给组合子演算写出一套重写系统:
\[\begin{aligned}
\cons{I}x & \rightsquigarrow x\\
\cons{K}xy &\rightsquigarrow x\\
\cons{S}xyz &\rightsquigarrow (xz)(yz)
\end{aligned}\]
那么设 \(\omega = \cons{S}\cons{I}\cons{I}\),
它对应 \(\lambda\)-演算中的 \(\omega = \lambda x. xx\),
因为由规约规则得到
\[\omega x = \cons{S}\cons{I}\cons{I}x \rightsquigarrow (\cons{I}x)(\cons{I}x) \rightsquigarrow xx.\]
请读者动笔规约表达式 \(\Omega = \omega\omega\).
如果计算无误就会发现, \(\Omega\) 无论如何规约, 几步之后
就会回到自己. 因此 \(\Omega\) 无法规约到既约形式.
换句话说, 无类型的组合子演算不是弱停机的, 从而也不可能是
强停机的.

用重写系统处理简单类型组合子与 \(\lambda\)-算术,
最终可以证明它们是强停机的, 参见~\cite{loader:1998:stlc}.
然而, 许多规则难以用重写系统表述. 譬如我们希望
对于无序数对有 \(\{x,y\} = \{y,x\}\). 使用重写系统
处理这种等式显然有些困难. 当代研究的不少类型论无法使用
重写系统描述. 所以我们有必要考虑别的方式. Sterling
在\cite{sterling:2021:thesis}中考虑了更加广义的概念,
在英文中仍然使用了 normalization 一词, 我们改译
为\textbf{典范化}.

类型系统之间千差万别, 因此也极难对“典范化”一词做通用定义.
这里我们看几个例子.

\subsection{简单类型组合子演算的典范化}
在组合子演算中, 我们可以定义\textbf{典范形式}的概念.
典范化即为“一切表达式都等价于某个典范形式”. 我们定义
不含 \(\cons{S}ABC\) 或者 \(\cons{K}AB\) 的表达式
为典范的, 其中 \(A,B,C\) 为子表达式. 因此类似
\(\cons{S}(\cons{K}(\cons{K}x))\cons{S}\) 的表达式
是典范的.

我们类比简单类型 \(\lambda\)-演算的方案定义类型系统.
\[\cons{S} : (\alpha \to \beta \to \gamma)
\to (\alpha \to \beta) \to (\alpha \to \gamma)\]
\[\cons{K} : \alpha \to \beta \to \alpha.\]
其中 \(\alpha,\beta,\gamma\) 可以任意取值. 我们添加
一个类型 \(\mathbb B\) 作为基础类型, 有两个元素
\(\cons{y}, \cons{n} : \mathbb B\). 注意这个
类型中还有有其他元素, 如 \(\cons{K}\cons{y}\cons{n}\) 等.
但是在我们之前规定的运算的等价关系下, \(\cons{K}\cons{y}\cons{n}\) 等价于 \(\cons{y}\).
最后我们有规则
\[\frac{A : \alpha \to \beta \quad B : \alpha}{AB : \beta}\]
回忆分数线上方是前提, 下方是结论. 因为组合子演算中没有变量,
定义比 \(\lambda\)-演算方便许多. 直接对表达式长度归纳,
不难证明\(\mathbb B\)类型中的典范形式只有\(\cons{y}\)
与 \(\cons{n}\).

由于无类型组合子演算没有典范化的性质, 我们必然需要
在证明中利用类型. 因此我们可以对类型归纳: 如果命题
对 \(\alpha,\beta\) 均成立可以推出对 \(\alpha\to\beta\)
成立, 并且命题对 \(\mathbb N\) 成立, 那么命题就
对所有的类型均成立. 这种归纳法称为\textbf{结构归纳法}.
一般的归纳是沿着自然数的结构归纳, 而这里是沿着类型的结构归纳.

\begin{theorem}
简单类型组合子演算中, 一切表达式都有一个相等的典范
表达式.\footnote{之前定义的“相等”实际上是表达式之间的一个等价
关系, 如 \(\cons{K}xy\) 等价于 \(x\), 我们介绍时为了
方便写作 \(\cons{K}xy = x\). 它们作为表达式显然是不相同的.
而这里说的是一切表达式都有一个等价的典范表达式.}
\end{theorem}
\begin{proof}
对于任何类型 \(\alpha\), 设 \(R_\alpha\) 是这个类型中
可典范化的表达式, \(T_\alpha\) 为所有表达式. 我们需要
证明的就是 \(R_\alpha = T_\alpha\). 但是直接对 \(\alpha\)
归纳是无法完成证明的, 我们需要加强归纳假设.

对于基础类型 \(\mathbb B\), 我们设 \(M_{\mathbb B} = R_{\mathbb B}\).
对于函数类型 \(\alpha \to \beta\), 考虑
\[M_{\alpha \to \beta} = \{F : \alpha \to \beta \mid
\forall A \in M_\alpha, FA \in M_\beta\}.\]
这样 \(M_{\alpha \to \beta}\) 中的元素都对应
一个 \(M_\alpha \to M_\beta\) 函数. 可以发现这就是
一个Henkin模型. 这样定义的 \(M_\bullet\) 使得对类型
的归纳可以顺利进行. 读者可以自行对类型归纳验证 \(M_\alpha\)
均保持表达式上的等价关系. 对表达式长度归纳就可以证明
\(M_\alpha = T_\alpha\). 最后只需要证明 \(M_\alpha = R_\alpha\)
即可. 而这对类型归纳也很简单.
\end{proof}

可以注意到, 对表达式的长度归纳在这里实际上也是对表达式
的结构归纳法: 如果命题对 \(\cons{S}, \cons{K}\) 都
成立, 并且如果对 \(A,B\) 成立则对 \(AB\) 成立, 那么命题
对一切组合子表达式都成立.

\subsection{简单类型\texorpdfstring{\(\lambda\)}{Lambda}-演算的闭典范化}
\berry{2}
我们同样规定有一个基本类型 \(\mathbb B\), 含有两个
元素 \(\cons{y}, \cons{n}\). 简单类型 \(\lambda\)-演算
中可以含有自由变量, 为了严格说明这些自由变量的类型, 我们需要
引入\textbf{上下文}(context)的概念.
在简单类型的情况下, 上下文就是一些变量的列表, 附上每个
变量的类型. 因此上下文形如 \(x{:}\alpha, y{:}\beta, \dots, z{:}\gamma\).
我们用大写希腊字母表示上下文. 如果某个表达式 \(M\)
在上下文 \(\Gamma\) 中有类型 \(\alpha\),
那么我们写作 \(\Gamma \vdash M : \alpha\).

我们可以将类型规则用这个记号写出来:
\[\frac{(x{:}\alpha) \in \Gamma}{\Gamma \vdash x : \alpha}
\quad\frac{\Gamma, x{:}\alpha \vdash M : \beta}{\Gamma \vdash \lambda x^\alpha. M : \alpha \to \beta}
\quad\frac{\Gamma \vdash M : \alpha \to \beta
\quad \Gamma \vdash N : \alpha}{\Gamma \vdash MN : \beta}.\]
注意这里严格来说 \(\lambda x. M\) 实际上需要标注变量的类型,
但是一般情况下我们都省略了.

其次, 我们添加乘积类型. 有乘积类型的 \(\lambda\)-演算
有闭典范化性质, 显然意味着不使用乘积类型的子集也有闭典范化
性质. 而加入了乘积类型更能展现其本质结构.
\[
\frac{\Gamma \vdash M : \alpha \quad \Gamma \vdash N : \beta}{\Gamma \vdash (M, N) : \alpha \times \beta}
\quad\frac{\Gamma \vdash M : \alpha\times\beta}{\Gamma\vdash\pi_1(M) : \alpha}
\quad\frac{\Gamma \vdash M : \alpha\times\beta}{\Gamma\vdash\pi_2(M) : \beta}
\]
乘积类型就是有序对的类型. 我们有 \(\pi_1(M, N)\) 等价于
\(M\), \(\pi_2(M, N)\) 等价于 \(N\). 反过来, 任何有序对
\(M\) 都等价于 \((\pi_1(M), \pi_2(M))\). 这两个
等价关系与函数一样, 分别称作 \(\beta\)- 与 \(\eta\)-等价.

\textbf{闭典范化}
说的是任何一个不含自由变量的表达式
\(\vdash M : \mathbb B\) 都
等价于 \(\cons{y}, \cons{n}\) 其中之一.
而更强一些的\textbf{典范化}说的是任何一个表达式
\(\Gamma \vdash M : \alpha\) 都等价于某个典范表达式.
我们互相递归地定义典范表达式和中性表达式:
\begin{definition}
\textbf{典范表达式}要么是类型为 \(\mathbb B\) 的中性表达式,
要么是 \(\cons{y},\cons{n}\) 之一,
要么形如 \(\lambda x. M\) 或 \((M, N)\),
其中 \(M, N\) 是典范表达式.
\textbf{中性表达式}要么是变量, 要么形如 \(M(N)\) 或
\(\pi_i(M)\), 其中 \(M\) 是中性表达式, \(N\) 是典范表达式.
\end{definition}

(...) TODO more examples and motivate this def

例如 \(\lambda x^{\mathbb B \to \mathbb B}. x\)
不是典范表达式, 而 \(\lambda x^{\mathbb B \to \mathbb B}.
\lambda y^{\mathbb B}. x(y)\) 是与它等价的典范形式.

无论是证明闭典范化还是典范化的性质都需要不少功夫, 因为
与组合子的情况类似, 简单地直接归纳是行不通的.
我们需要先通过归纳法证明一个更强的
命题. 我们的大致思路仍然是在每个类型 \(\sigma\) 上定义一个谓词
\(R_\sigma\), 并归纳证明其性质. (因为集合\(X\)上
的谓词\(P\)等价于它的子集 \(\{x\in X\mid P(x)\}\), 所以
这两者可以交换使用.) 如
\[R_{\sigma \to \tau} = \left\{(\Gamma, f) \,\middle|\,
\begin{aligned}
\Gamma \vdash&\, f:\sigma\to\tau, \\ \forall (\Delta, s) \in R_\sigma, \Gamma \subseteq \Delta &\implies (\Delta, f(s)) \in R_\tau
\end{aligned}
\right\}.\]
这比较不直观, 而且很难想到.
利用范畴语言, 可以给出更加系统的表述, 其中这些略微复杂的
构造其实都是自然而唯一的.

为此, 我们首先需要将简单类型\(\lambda\)算术的语法做成
范畴语言. 我们令所有的上下文 \(\Gamma\) 为范畴
\(\mathcal T\) 中的对象, 而 \(\Gamma\) 到 \(\Delta\)
的态射就是在 \(\Gamma\) 上下文中的一列表达式, 其中第
\(n\) 个表达式的类型恰好是 \(\Delta\) 中的第 \(n\) 个
类型. 换言之, 这一列表达式给出了从 \(\Delta\) 到 \(\Gamma\)
的\textbf{替换} (substitution). 并且我们认为
等价的表达式给出相同的替换.

(...) More examples, perhaps mention substitutions earlier.
empty is terminal

我们实际上有非常好的方式刻画这个范畴. 我们已经介绍过了
范畴中的Descartes乘积. 而函数也有范畴定义.
\begin{definition}
Exponential object
\end{definition}
\begin{definition}
CCC, CCFunctor, what is a good translation?
\end{definition}
\begin{theorem}\label{beginning:lambda:initial}
\(\mathcal T\) 是由图表 \(1 \rightrightarrows \mathbb B\)
生成的自由积闭范畴.
\end{theorem}
“自由生成”在这里的含义或许需要阐明. 我们定义这样一种数学
结构, 其中包含一个积闭范畴以及选定一个对象 \(A\)
和两个态射 \(1\rightrightarrows A\). 这些数学结构之间
可以定义一些同态, 即保持选定的对象和态射的保积闭函子.
那么这就构成了一个范畴\footnote{explain 2-cat}, 姑且称作
\(\cons{CartClosedCat}_{1\rightrightarrows\mathbb B}\).
那么其中的始对象就是我们所说的被自由生成的范畴.

\begin{proof}
把这个定理展开, 就可以发现它整齐地打包了很多预备的引理.
在这里我们只提到重要的引理, 其余部分都是直接展开定义即可.
\begin{itemize}
\item 首先我们需要证明 \(\mathcal T\) 构成范畴, 需要验证
结合律. 这就是\textbf{替换引理}. (很不幸的是
这个词可以指代两个不同的引理.) 这可以通过对表达式的结构做
归纳得到.
\item 其次我们需要证明 \(\mathcal T\) 是保积闭函子. 这实际上就是
简单类型\(\lambda\)-算术的语法决定的, 而其中那些自然同构
的等式则是由 \(\beta,\eta\) 规则保证.
\item 我们接下来需要证明, 对于任意一个这样的范畴 \(\mathcal C\),
都能找到保持选定的对象和态射的保积闭函子
\(F : \mathcal T \to \mathcal C\). 这之中需要证明
的是替换与\(\beta\eta\)-等价是可交换的. 而这函子
的唯一性来源于其所有类型都是用
\(\mathbb B, \times, \to\) 三者生成的.\qedhere
\end{itemize}
\end{proof}

一个\textbf{范畴模型}就是从语法范畴出发的保持语法结构的函子.
在我们这里, 简单类型\(\lambda\)理论的模型就是从
语法范畴 \(\mathcal T\) 出发的保持
\(\mathbb B,\cons{y},\cons{n}\)
的保积闭函子. 回顾上文, 就会发现我们刚刚已经证明的
定理~\ref{beginning:lambda:initial} 可以表述
成 “\(\mathcal T\) 是简单类型\(\lambda\)-演算
的范畴模型构成的2-范畴中的始对象”. 特别地, 语法范畴
本身也是模型之一.

我们现在就可以迅速证明 \(\cons{y}\ne\cons{n}\).
考虑所有集合的范畴 \(\cons{Set}\), 它是积闭的.
我们再选择 \(1 \rightrightarrows A\) 为
\(1=\{\varnothing\}\) 到
\(2=\{\varnothing, \{\varnothing\}\}\) 的两个态射即可.
如果 \(\cons{y}=\cons{n}\),
那么说明定理~\ref{beginning:lambda:initial} 产生的函子
\(\mathrm s:\mathcal T \to \cons{Set}\)
满足 \(\mathrm{s}(\cons{y}) = \mathrm{s}(\cons{n})\),
进一步得到集合 \(2\) 的两个元素相等, 矛盾. 注意这个模型
仍然无法证明典范性, 因为 \(\mathrm{s}\) 仍然有可能
把不相等的表达式 \(M,N \in
\hom_{\mathcal T}(1, \mathbb B)\)
映射到相等的函数.

为了研究典范性, 我们需要先找出类型论中的闭元素, 即不包含
自由变量的表达式. 这很容易, 我们只需要要求上下文为空即可.
因此我们考虑 \(\boldsymbol\Gamma(X) = \hom_{\mathcal T}(1, X)\).
这是一个 \(\mathcal T \to \cons{Set}\) 的函子.

\begin{comment}
如上文所述, 我们一般的思路是为每一个类型赋予一个谓词.
在范畴论中, 我们经常使用一个 \(f : Y \to X\) 的态射作
为\(X\)上的“谓词”. 大致来说, 我们认为如果 \(f^{-1}(x)\) 非空,
那么这个谓词就取值为真, 否则为假. (当然这个大致说法只在
集合范畴里成立, 其他范畴中只能起启发作用.)\footnote{有些
读者可能会认为, 我们需要让 \(f\) 为单射,
即 \(f^{-1}(x)\) 要么是空集要么是单点集. 在这里这个要求
可有可无. 没有这个要求时行文更简洁.} 因此, 我们考虑这样
一个范畴 \(\mathcal G\), 它的对象是
形如 \(f:\Sigma' \to \boldsymbol\Gamma(\Sigma)\) 的态射,
(准确地说, 是一个三元组 \((\Sigma', f, \Sigma)\).)
其中 \(\Sigma\) 是 \(\mathcal T\) 中的对象, \(\Sigma'\)
是集合. 这个范畴里的态射是这样的交换图:
\[
    \begin{cd}
        \Sigma' \ar[r, "\sigma'"] \ar[d] & \Delta' \ar[d] \\
        \boldsymbol\Gamma(\Sigma) \ar[r, "\boldsymbol\Gamma(\sigma)"] & \boldsymbol\Gamma(\Delta)
    \end{cd}
\]
其中 \(\sigma\) 是 \(\Sigma \vdash \Delta\) 的
一个替代, \(\sigma'\) 是某个函数.

如果读者熟悉范畴论, 就应该知道这是[[逗号范畴]]的特例:
\[\mathcal G = \mathrm{Id}_{\cons{Set}}\downarrow \boldsymbol\Gamma.\]
它也可以看成是下面这个拉回:
\[
    \begin{cd}
        \mathcal G \ar[r, dashed] \ar[d, dashed, "\mathrm{gl}"'] \ar[dr, phantom, "\lrcorner", very near start] & \cons{Set}^{\to} \ar[d, "\mathrm{cod}"] \\
        \mathcal T \ar[r, "\boldsymbol\Gamma"'] & \cons{Set}
    \end{cd}
\]

注意这时候我们自动有了一个函子 \(\mathrm{gl}\), 它把
\(\mathcal G\) 中的对象 \(\Sigma' \xrightarrow f \boldsymbol\Gamma(\Sigma)\)
映射到 \(\Sigma\).

我们希望能够取 \(\mathcal G\) 为一个模型. 回顾模型的定义,
我们的第一步当然是要证明:
\begin{lemma}
\(\mathcal G\) 是 Descartes 闭范畴, 并且 \(\mathrm{gl}\)
是 Descartes 闭函子.
\end{lemma}
这个证明实际上对应的就是通常的证明中在类型上递归定义谓词.
但是注意这个命题本身已经完全限制了构造的可能性, 因此无需
任何创意, 仅仅是机械地展开定义即可. 留给读者作为练习.

接下来, 我们只需要在 \(\mathcal G\) 中选择一个对象
\(A\) 和两个态射 \(1 \rightrightarrows A\). 选择了
之后, 根据定理\ref{initial}我们立刻得知有唯一的保持
这个对象和态射的函子 \(\mathrm{s}:\mathcal T \to \mathcal G\).
如果 \(\mathrm{gl}\) 也保持这个对象和态射的话, 那么
这两个函子都在 \(\cons{CartClosedCat}_{1\rightrightarrows\cons{Ans}}\)
中. 根据始对象的性质, \(\mathrm{gl} \circ \mathrm{s} : \mathcal T \to \mathcal T\)
必然等于恒等函子 \(\mathrm{Id}_{\mathcal T}\). 知道
了这个, 我们自然要选择合适的对象. 这也不难, 如图构造即可:
\[
    \begin{cd}
        1 \ar[r, "\cons{y}"] \ar[d] & \{\cons{y}, \cons{n}\} \ar[d, hook] & \ar[l, "\cons{n}"'] 1 \ar[d] \\
        \boldsymbol\Gamma(1) \ar[r, "\boldsymbol\Gamma(\cons{y})"'] & \boldsymbol\Gamma(\cons{Ans}) &\ar[l, "\boldsymbol\Gamma(\cons{n})"] \boldsymbol\Gamma(1)
    \end{cd}
\]
注意整个交换图都在 \(\cons{Set}\) 中.

现在 \(\mathrm{gl}\circ\mathrm{s}\) 等于恒等函子.
我们需要证明的是任何一个表达式 \(M \in \boldsymbol\Gamma(\cons{Ans})
= \hom_{\mathcal T}(1, \cons{Ans})\) 都一定恰好是
\(\cons{y},\cons{n}\) 其中之一.
因此我们考虑 \(\mathrm{s}(M) \in \hom_{\mathcal G}(\mathrm{s}(1), \mathrm{s}(\cons{Ans}))\),
根据 \(\mathcal G\) 中态射的定义, 这对应一个交换图:
\[
    \begin{cd}
        1 \ar[r, dashed, "M'"] \ar[d] & \{\cons{y}, \cons{n}\}\ar[d]\\
        \boldsymbol\Gamma(1) \ar[r, "\boldsymbol\Gamma(M)"'] & \boldsymbol\Gamma(\cons{Ans})
    \end{cd}
\]
其中左侧是 \(1 \to \boldsymbol\Gamma(1)\),
因为 \(\mathrm{s}\) 是 Descartes 闭函子,
因此在同构意义下保持始对象. 右侧则是因为 \(\mathrm{s}\)
按照定义需要保持范畴中选定的 \(A\) 对象. 那么虚线
表示的映射就选出了 \(\cons{y}, \cons{n}\)
的其中一个. 无论是哪种, \(\mathrm{gl}\) 必然保持这个映射.
又因为 \(\mathrm{gl}(\mathrm{s}(M)) = M\),
我们可以得出结论, \(M\) 必然是 \(\cons{y}, \cons{n}\)
其中之一.\footnote{直接观察交换图, 并且对 \(\mathbf{\Gamma}\)
做一些分析, 可以绕过 \(\mathrm{gl}\circ\mathrm{s} = \mathrm{Id}\) 直接说明
这一点, 不过使用 \(\mathrm{gl}\) 说理在后面更加干净.}
\end{comment}

\subsection{简单类型\texorpdfstring{\(\lambda\)}{Lambda}-演算的典范化}\berry{4}
Kripke 语义, normalization

\section{论域论}\label{beginning:domain}\berry{1}

这一节主要参考~\cites{abramsky:1995:domain}{cartwright:2016:domain}.
尽管无类型的 \(\lambda\)-演算与组合子演算不能直接用于
逻辑, 但是仍然可以用来表示计算. 因此寻找它的一套语义表示
也是有意义的. 一个非常重要的观察是, 无类型的本质是\textbf{单类型}.
换句话说, 实际上所有的表达式都有一个相同的类型. 这个
观察在其他一些地方会非常有用. 在这里, 我们需要有一个函数
类型 \(\alpha \to \beta\), 但是一切的类型都相同, 所以
我们就顺理成章地得到了一个等式:
\[(D \to D) \cong D.\]
因此, 无类型 \(\lambda\)-演算的语义大致上是要找到一个数学
对象 \(D\), 使得 \(D \to D\), 也就是某种函数或者映射
的空间, 满足这个同构. 其中我们的 \(\lambda\)
语法是从左往右的对应, 而函数求值则是从右往左的对应.

但是这种对象并不常见: 如果 \(D\) 是集合, 那么函数集
\(D \to D\)
和 \(D\) 有双射, 当且仅当 \(D\) 只有一个元素. 因为如果
\(D\) 有至少两个元素, 那么
\(D^D \ge 2^D > D\).
这里第二个不等式来源于 Cantor 的对角线论证.
而只有一个元素的集合过于平凡.

但是, 其他数学对象仍然是有机会的. 譬如假设 \(D\) 是一个
拓扑空间, 那么 \(D \to D\) 是连续函数构成的空间. 它至少
在集合元素数量上是可以和 \(D\) 有双射的. 那么能不能构造
一个拓扑空间使得 \(D \to D\) 和 \(D\) 之间有同胚呢?
1960年代, Dana Scott 为了解决这个问题, 提出了一系列数学对象, 主要
来自于拓扑空间, 以及各种各样的序(order)和格(lattice)结构.
研究这些结构的理论被称为\textbf{论域论}(domain theory).

我们不做过多详细的讨论, 而是简单介绍一下这些数学对象的直观.
\begin{definition}[{\cite[\S\S2.1.5--2.1.6]{abramsky:1995:domain}}]
给定一个集合 \(D\) 上的偏序 \(\succeq\), 它
的\textbf{定向子集}(directed subset)为那些满足对于所有 \(x,y\in A\),
存在 \(z \in D\) 使得 \(z \succeq x, y\) 的非空子集
\(A\). 如果每个定向子集 \(A\) 的上确界都存在, 则称其
为\textbf{定向完备偏序}(directed-complete partial order, 简写dcpo).
两个dcpo之间的\textbf{Scott-连续函数}为满足
\[f\left(\sup A\right) = \sup f(A)\]
的单调函数. 其中 \(A\) 取遍所有的定向子集. 注意单调函数满足
定向子集的像仍然是定向子集.
\end{definition}
我们可以给dcpo赋予一种拓扑, 称为 \textbf{Scott拓扑},
使得 Scott-连续函数就是在这些拓扑下连续的函数.
Scott-连续函数的集合 \([A \to B]\) 逐点使用 \(B\) 上的
偏序可以诱导出一个偏序. 在这个偏序下定向子集也都有上确界,
即逐点的上确界.

我们还可以在dcpo上添加各种各样的要求, 比如有最小元 \(\bot\),
有连续性, 代数性等等; 就像环论中我们也会研究满足各种条件
的环. 这些空间统称为\textbf{论域}(domain).\footnote{和哲学或者逻辑学中的论域不同.}
而大致来说, 这些空间在求解类似 \([D \to D] \cong D\),
或者更一般的 \(F(D) \cong D\) 方程时都表现了优良的性质.
我们可以找到最小的\(D\)满足这样的方程. 具体在 \(\lambda\)-演算
上, 我们得到的偏序中每个点表示一组信息, 而在偏序中越大表示
信息越详细. 而上面所说的 \(\Omega = (\lambda x. xx) (\lambda x.xx)\)
则对应最小元 \(\bot\). 我们甚至可以给出 \(\cons{Y}\)
组合子的一种刻画.
\begin{lemma}[{\cite[定理2.1.19]{abramsky:1995:domain}}]
对于有最小元 \(\bot\) 的dcpo \(A\),
存在Scott-连续函数 \[\cons{fix} : [A \to A] \to A\]
取出每个函数的最小不动点.
\end{lemma}
\begin{proof}
对于每个Scott-连续函数,
考虑数列 \(\bot, f(\bot), f(f(\bot)),\dots\).
由最小元的定义有 \(\bot \preceq f(\bot)\). 再反复使用
\(f\) 的单调性得到 \(f^{(n)}(\bot) \preceq f^{(n+1)}(\bot)\).
因此这是一条升链, 从而是定向子集. 它有上确界
\(f^{(\infty)}(\bot) = \sup_{n\in\mathbb N}f^{(n)}(\bot).\)
那么由 \(f\) 的 Scott 连续性有
\[f(f^{(\infty)}(\bot)) = f(\sup_{n\in\mathbb N}f^{(n)}(\bot))
= \sup_{n\in\mathbb N} f(f^{(n)}(\bot)) = f^{(\infty)}(\bot).\]
故这的确是不动点, 并且不难看出这是最小的不动点.
另一方面, 考虑 \(n\) 次迭代操作 \(f \mapsto f^{(n)}(\bot)\),
这是Scott-连续函数, 从而这些迭代函数对 \(n\) 取逐点上确界
也是Scott-连续的, 所以 \(\cons{fix}\) 是连续函数.
\end{proof}

我们构造出满足 \([D \to D] \cong D\) 的对象后,
可以证明\(\cons{Y}\)的语义就是 \(\cons{fix}\) 函数.
回忆我们认为dcpo中元素越大表示信息越详细.
换句话说, 尽管
某个函数可能有很多不动点, 但是 \(\cons{Y}\)
\emph{总是取出定义最“不明确”的那个}. 譬如 \(\lambda x.x\)
的不动点为全体表达式. 而 \(\cons{Y}(\lambda x. x) = \Omega\),
选出了信息“最少”的那个, 即 \(\bot\).

论域论不仅提供了一种框架, 还将计算机科学、数学与逻辑学联系
起来. 特别是计算机科学中经常需要处理不停机的递归, 以及
集合论难以处理的函数. 最经典的例子就是下一章马上会介绍的
F 系统, 而不少编程语言正是以 F 系统为蓝本构建的.
