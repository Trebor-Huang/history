\documentclass[UTF8, zihao=-4]{ctexart}
\usepackage[T1]{fontenc}
\usepackage[a4paper]{geometry}
\usepackage{graphicx}
\usepackage[dvipsnames]{xcolor}
\usepackage{amssymb,amsmath,amsthm}
\usepackage{fancyhdr}
\usepackage{biblatex}
\usepackage{hyperref}

\newcommand\myshade{85}
\colorlet{mylinkcolor}{violet}
\colorlet{mycitecolor}{orange}
\colorlet{myurlcolor}{Aquamarine}

\hypersetup{
  linkcolor  = mylinkcolor!\myshade!black,
  citecolor  = mycitecolor!\myshade!black,
  urlcolor   = myurlcolor!\myshade!black,
  colorlinks = true,
}

\newcommand{\cons}[1]{\textsf{#1}}
\newcommand{\slogan}[1]{\begin{center}%
  \fcolorbox{red!80!green}{red!10}{%
    \large\textbf{#1}%
  }%
\end{center}}  % TODO change later

\theoremstyle{plain}
\newtheorem{lemma}{引理}
\newtheorem{theorem}{定理}
\newtheorem{corollary}{推论}

\theoremstyle{definition}
\newtheorem{definition}{定义}

\theoremstyle{remark}
\newtheorem*{remark}{注}

\addbibresource{history.bib}

\pagestyle{fancy}
\fancyhead[L]{\nouppercase{\kaishu\rightmark}}
\fancyhead[R]{\nouppercase{\kaishu\leftmark}}
\setlength{\headheight}{14pt}
\title{类型论简史 (摘要)}
\date{}
\author{Author}
\begin{document}
\maketitle

2020年12月, Peter Scholze发布了\href{https://xenaproject.wordpress.com/2020/12/05/liquid-tensor-experiment/}{一项挑战}.
在他的凝聚态数学前沿研究中有一条技术性定理, 对于这个领域有
至关重要的作用:
\begin{theorem}[Clausen, Scholze]
设 \(0 < p' < p \le 1\) 为实数, 令 \(S\) 为投射有限集,
\(V\) 为 \(p\)-Banach 空间. 记 \(\mathcal M_{p'}(S)\)
为 \(S\) 上的 \(p'\)-测度构成的空间. 那么对于 \(i \ge 1\) 有
\[\mathrm{Ext}_{\mathsf{Cond}(\mathsf{Ab})}^i(
  \mathcal M_{p'}(S), V
) = 0.\]
\end{theorem}
Scholze希望这条定理的证明可以被完全严格形式化, 并且通过
计算机的检查. 2022年7月, Lean
社区\href{https://leanprover-community.github.io/blog/posts/lte-final/}{宣布}这个挑战正式完成.
换句话说, 即使是数学最尖端的结论, 也可以在一年半的时间内
被转化成计算机可以理解并验证的代码.
这种成就是如何达成的呢? 答案是Lean的类型论.

在类型论中, 一切数学对象的含义都由它们从属的类型决定.
如类型 \(\mathbb N\) 的元素是自然数, 而类型 \(\alpha \times \beta\)
的元素是有序对, 类型 \(\alpha \to\beta\) 的元素
是映射, 等等. 而这些类型可以相互组合, 表达出复杂的含义,
这使得类型论有能力作为数学的基础, 与集合论的地位类似.
另一方面, 计算机科学中也有利用类型描述程序的传统. 因此, 类型论
可以看作是数学与计算机之间的桥梁. 这也使得计算机定理
验证成为了有可行性的工作: 传统的使用一阶逻辑与集合论数学
基础中, 一个初等的定理也可能需要冗长的说明才能完全
严格的写出来, 因此只有理论上的可能性.
类型论的定理验证不仅可以验证数学定理,
在实际应用上还可以验证各种需要低出错率的
软硬件(如军事、医用、宇航电子仪器)设计无误.


这篇文章的主要目的是以历史作为连贯的主线将类型论的知识
串联在一起, 为中文学习者提供些微的参考价值. 同时, 文章中
介绍各个领域时都会提及其中优秀的参考书目, 供希望进一步了解
的读者阅读.

文章不要求任何类型论的前置知识, 并且尽量避免范畴论知识的
使用, 但是会援引一般数学中的许多例子. 同时, 每一节的内容
相对独立, 因此如果某一部分使用的范畴论知识较多, 读者可以直接
跳过, 不影响后续阅读.

Russell 悖论, 简单类型论

Curry--Howard 对应

MLTT, HoTT, CuTT

类型论的语义

\end{document}